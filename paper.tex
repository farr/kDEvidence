\documentclass[prd,preprint]{revtex4}

\usepackage{amsmath}
\usepackage{hyperref}

\newcommand{\vtheta}{\vec{\theta}}

\bibliographystyle{h-physrev}

\begin{document}
\title{An Efficient Interpolation Technique for Jump Proposals in
  Reversible-Jump Markov Chain Monte Carlo Calculations}

\date{\today}

\author{Will M. Farr}
\email{w-farr@northwestern.edu}

\affiliation{Northwestern University Center for Interdisciplinary
  Research and Education in Astrophysics}

\author{Ilya Mandel}
\email{ilyamandel@chgk.info}

\affiliation{MIT Kavli Institute}

\begin{abstract}
  Reversible-jump Markov chain monte carlo (RJMCMC) is an extremely
  powerful technique for performing Bayesian model selection, but it
  suffers from a fundamental difficulty: it requires jumps between
  model parameter spaces, but cannot retain a memory of the favored
  locations in more than one parameter space at a time.  Thus, a naive
  jump between parameter spaces is unlikely to be accepted in the MCMC
  algorithm and convergence is correspondingly slow.  Here we
  demonstrate an interpolation technique that uses samples from
  single-model MCMCs to propose inter-model jumps from an
  approximation to the single-model posterior of the target parameter
  space.  The interpolation technique, based on a kD-tree
  datastructure, is adaptive and efficient in arbitrary dimensions.
  We show that our technique leads to dramatically improved convergene
  over naive jumps in an RJMCMC, and compare it to other proposals in
  the literature to improve the convergence of RJMCMCs.  We also
  discuss the use of the same interpolation technique in two other
  contexts: as a convergence test for a single-model MCMC and as a way
  to construct efficient ``global'' proposal distributions for
  single-model MCMCs without prior knowledge of the structure of the
  posterior distribution.
\end{abstract}

\maketitle

\section{Introduction}

\section{Reversible Jump MCMC}

Reversible jump MCMC (RJMCMC) \cite{Green1995} is a technique for
Bayesian model comparison.  Consider a set of models indexed by an
integer, $i$: $\{M_i | i = 1, 2, \ldots \}$.  Each model has some
continuous parameters, $\vtheta_i$; given the model and its
parameters, we can make a prediction about the likelihood of observing
some experimental data, $d$: $L(d|\vtheta_i, M_i)$.  When some data
have been observed in an experiment, Bayes' rule gives us a way to
compute the posterior probability distribution for the model
parameters implied by the data:
\begin{equation}
  p(M_i, \vtheta_i | d) = \frac{L(d|\vtheta_i, M_i) p(\vtheta_i|M_i) p(M_i)}{p(d)},
\end{equation}
where $p(M_i, \vtheta_i |d)$ is the posterior distribution for the
model parameters $\vtheta_i$ implied by the data in the context of
model $M_i$, $p(\vtheta_i|M_i)$ and $p(M_i)$ are the prior
probabilities of the model parameters and the model itself that
represent our beliefs before accumulating any of the data $d$, and
$p(d)$ is an overall normalizing constant that ensures that $p(M_i,
\vtheta_i|d)$ is properly normalized as a probability distribution on
the $M_i$ and $\vtheta_i$.  This implies that 
\begin{equation}
  p(d) = \sum_i \int_{V_i} d\vtheta_i p(M_i, \vtheta_i|d) \equiv
  \sum_i p(M_i|d),
\end{equation}
where $V_i$ is the parameter space volume in model $M_i$, and we have
defined the integrated model posterior, 
\begin{equation}
  p(M_i|d) \equiv \int_{V_i} d\vtheta_i p(M_i, \vtheta_i | d),
\end{equation}
which represents the total posterior probability in the model $M_i$
parameter space.





\section{kD Trees and Interpolation}

\section{RJMCMC Efficiency}

\section{Example}

BH mass paper [still not sure how to present this without repetition...  ]  




\section{Conclusion/Discussion}

...

\subsection{Other Uses}
The efficient interpolation of a PDF via a kD-tree described in this paper can be extremely useful in other contexts, beyond a reversible-jump MCMC.  We suggest two examples below: generating efficient jump proposal distributions and convergence tests?

...

Mention nested sampling?

\nocite{Littenberg2009}

\bibliography{paper}

\end{document}